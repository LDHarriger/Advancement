\mychapter{4}{Proposed work}
%
% The proposal should address the feasibility of various experiments and point out caveats that might be encountered and how these could be circumvented. Be sure to include positive and negative controls, analysis and interpretation, pitfalls and alternative approaches, and somewhat detailed methods. Prioritize
%

%%%%%%%%%%%%%%%%%%%%%%%%%%%%%%%%%%%%%%%%%%%%%%%%%%%%%%%%%%%%%%%%%%%%%%%%%%%%%%%%%%
%%%%%%%%%%%%%%%%%%%%%%%%%%%%%%%%%%%%%%%%%%%%%%%%%%%%%%%%%%%%%%%%%%%%%%%%%%%%%%%%%%
\section*{Name-face association task}
To investigate the effects of sleep on memory consolidation, I propose creating an audio-visual name-face association task administered with an intervening period of sleep between training and testing phases. While characterizing neural processes underlying name-face association are interesting in there own right and have not been explored, I have designed the task more specifically to examine how neural processes during sleep facilitate this learning. Additionally, we will assess whether auditory stimulation during sleep further supports memory consolidation. Ultimately, I intend to test the hypothesis that reactivation of task related network dynamics underlie memory consolidation during sleep.

\subsection{Background}
Auditory stimulation has been shown to evoke similar neural activity as auditory processing during wake \citep{Sela2016b, Makov2017e}. Likewise, there are various reports that such stimulation during sleep improve memory \citep{Sela2016b, Makov2017e}.

While there are many reports of replay events during sleep in rodents, evidence is lacking in humans. Early research on replay was generally presented in the context of the hippocampus and multi-unit firing patterns, but memories are also coded in distributed cortical networks. Lately, more emphasis has been placed on the interactions between the hippocampus and cortex which are believed underlie memory consolidation, and occur frequently during sleep. Specifically, these consolidation processes are though to involve cortical SWs, thalamic spindles, and hippocampal SWRs. A recent human iEEG study provided the first purported evidence of replay in humans by demonstrating that sequences of HG peaks related to task activity are repeated during sleep \citep{Jiang2017}. Nevertheless, this study was only really only suggestive, relating spontaneous sequences of activity during wake to those during sleep, and in three subjects, during varied cognitive tasks. To conclusively show evidence of replay, it is necessary to show that unique task-induced activity patterns are reactivated, and that such reactivation improves memory performance on the task.

To that end, we have designed a task with rich audio-visual stimuli pairs.
%hippocampal dependent learning?
%%%%%%%%%%%%%%%%%%%%%%%%%%%%%%%%%%%%%%%%%%%%%%%%%%%%%%%%%%%%%%%%%%%%%%%%%%%%%%%%%%
%%%%%%%%%%%%%%%%%%%%%%%%%%%%%%%%%%%%%%%%%%%%%%%%%%%%%%%%%%%%%%%%%%%%%%%%%%%%%%%%%%
%
% 
\subsection*{Stimuli}
The task involves learning associations between audio clips of a speaker introducing themselves and the picture of a face paired with that name. Names will be drawn from the social security database of names listed by frequency in the US and California. The audio of names will be spoken in a standard American English accent also with a neutral tone, and will be either recorded by voice actors or clipped from existing audio - for example on YouTube. Face stimuli will be high-quality color static images of people expressing a neutral affect, and featuring only the shoulders and above. The face will be set in front of a light backdrop and the images will be presented on a black background. Finally, any text will be presented with standard sans-serif font in white.  
% (https://www.ssa.gov/oact/babynames/)
\subsection*{Training}
During the training phase, subjects will be presented with 50 face-name pairs. Each presentation will be preceded by a white fixation cross on a black background for 100 ms. A face will then be presented, followed 500 ms later by an audio clip of the corresponding name. The face will remain on display for an additional 500 ms before it is followed by a screen prompting the subject to either (1) "repeat the name in your head" or (2) "imagine the face." The subject will be given 1000 ms before the next blank screen. 

\subsection*{Rehearsal}
%either the name or the face will be presented followed by a prompt 500 ms later. If the face is presented, the subject will be prompted to "say this person's name aloud" or (2) "say this persons name in your head." If the name was presented, the person will be prompted to "imagine this person's face." In either case, 1500 ms later, the correct name or face will follow. Rehearsal trials are designed to elicit memory reactivation, and by using varying prompts, we hope to identify how reactivation networks are recruited more generally.

\subsection*{Test}
A test phase will be administered immediately following training and again after sleep. During this phase, a face will be presented followed by a prompt to "say this person's name aloud" 500 ms later. No feedback will be given to indicate whether the subject was correct.

\subsection*{Sleep}
Before the final test session, subjects will be asked to sleep. Prior to them sleeping, a subset of five names will be selected to be played during sleep. A volume will be chosen that is slightly below that of casual speech so that the patient will not be woken.
% Random subjects or single unit response? Random played or cued by SW?
\cite{Sharon2017}

\subsection*{Subjects}
Although this task was designed for patients implanted with iEEG to probe fast network dynamics supporting to memory consolidation and reactivation during sleep, it need not be limited to this population. With few modifications it can be implemented for other populations to be studied using MRI and/or EEG. In fact, this task would be well suited to study AD patients, as memory for face-name associations is something that was previously shown to be predictive of amyloid burden.

\subsection*{Analyses}
Hypothesize task response will involve activity in near the superior lateral temporal lobe during name stimuli presentation. Then during face stimuli presentation, occipital lobe and fusiform gyrus activity will increase. We expect each of these activities to exhibit functional connectivity to the hippocampus and frontal lobe during imagined reactivation and.

\subsection*{Discussion}
The major Some advantage of this task are its wide range of  Single units
Memory alzheimer's
Biographical memories
Interpersonal memories
Emotion - voice or expression.

%%%%%%%%%%%%%%%%%%%%%%%%%%%%%%%%%%%%%%%%%%%%%%%%%%%%%%%%%%%%%%%%%%%%%%%%%%%%%%%%%%
%%%%%%%%%%%%%%%%%%%%%%%%%%%%%%%%%%%%%%%%%%%%%%%%%%%%%%%%%%%%%%%%%%%%%%%%%%%%%%%%%%
\section*{Evaluating Glymphatic Activity in Humans}
%%%%%%%%%%%%%%%%%%%%%%%%%%%%%%%%%%%%%%%%%%%%%%%%%%%%%%%%%%%%%%%%%%%%%%%%%%%%%%%%%%
%%%%%%%%%%%%%%%%%%%%%%%%%%%%%%%%%%%%%%%%%%%%%%%%%%%%%%%%%%%%%%%%%%%%%%%%%%%%%%%%%%

\subsection*{Subjects}
With ongoing collaboration with the ADRC, I have access to a large cohort of patients with Alzheimer's Disease willing to participate in MRI scans.

\subsection*{Flow MRI}
Preliminary analyses suggest that. 

\subsection*{Analysis}
Our application of PCMRI is specifically targeted to test the hypotheses that CSF flow into and out of brain parenchyma increases during slow wave sleep relative to wake. We expect the flow increase immediately surrounding arteries to have an increase oriented orthogonally from blood flow and into parenchyma, and vice-versa near the veins. We hypothesize these flows will occur rhythmically modulated by the cardiac cycle and respiratory cycle for the periarterial and perivenous voxels. Additionally, compared to healthy controls and patients with disease, we expect this CSF flow to be disrupted, both slower and less directed.

A successful execution would enhance our basic model for how brain rhythms are coordinated in space and how this coordination supports memory. Additionally, we evidence of glymphatic clearance.

\subsection*{Discussion}




%%%%%%%%%%%%%%%%%%%%%%%%%%%%%%%%%%%%%%%%%%%%%%%%%%%%%%%%%%%%%%%%%%%%%%%%%%%%%%%%%%
%%%%%%%%%%%%%%%%%%%%%%%%%%%%%%%%%%%%%%%%%%%%%%%%%%%%%%%%%%%%%%%%%%%%%%%%%%%%%%%%%%
% TRASH
%%%%%%%%%%%%%%%%%%%%%%%%%%%%%%%%%%%%%%%%%%%%%%%%%%%%%%%%%%%%%%%%%%%%%%%%%%%%%%%%%%
%%%%%%%%%%%%%%%%%%%%%%%%%%%%%%%%%%%%%%%%%%%%%%%%%%%%%%%%%%%%%%%%%%%%%%%%%%%%%%%%%%
%\subsection*{Audiovisual learning}
% Ears. The conical shape of the ear helps to focus air pressure waves onto the eardrum. These vibrations in turn displace a system of three bone -- the malleus, incus, and stapes -- which has the effect of amplifying this signal concentrating this force onto the smaller area of the stapes which beats against the fluid filled chamber of the cochlea. The vibrations of the cochlear fluid propagate through its spiral shape which helps to separate the sound into its frequency components: low frequencies are able to travel the entire length of the spiral, whereas higher frequencies travel shorter distances. Hair cells are located along this chamber, and the vibration of these hairs is finally transduces the original mechanical signal into action potentials which are carried to the brain stem by the auditory nerve. [Brain stem pathway]
%
%Eventually, these signals reach the auditory cortex, where the sounds are believed to be perceived as tones of various frequencies. This sort of fourier sound representation then propagates to other areas to endow the stimulus with more complex perceptions; like the visual system, there are believed to be dorsal and ventral processing streams related to the perception of where and what the stimulus is, respectively.

%%% Local Variables: ***
%%% mode: latex ***
%%% TeX-master: "thesis.tex" ***
%%% End: ***
