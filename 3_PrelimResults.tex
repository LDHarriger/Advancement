\mychapter{3}{Preliminary results}
%
% This section should include your research efforts. Appropriate discussion and methods are important; you should show how you can perform all of the necessary techniques and methods. Please embed figures into the text and include a brief legend. Figures and Tables must be absolutely clear and visible. 
%
%%%%%%%%%%%%%%%%%%%%%%%%%%%%%%%%%%%%%%%%%%%%%%%%%%%%%%%%%%%%%%%%%%%%%%%%%%%%%%%%%%%%
%%%%%%%%%%%%%%%%%%%%%%%%%%%%%%%%%%%%%%%%%%%%%%%%%%%%%%%%%%%%%%%%%%%%%%%%%%%%%%%%%%%%
\section*{Methods}
%%%%%%%%%%%%%%%%%%%%%%%%%%%%%%%%%%%%%%%%%%%%%%%%%%%%%%%%%%%%%%%%%%%%%%%%%%%%%%%%%%%%
%%%%%%%%%%%%%%%%%%%%%%%%%%%%%%%%%%%%%%%%%%%%%%%%%%%%%%%%%%%%%%%%%%%%%%%%%%%%%%%%%%%%

%%%%%%%%%%%%%%%%%%%%%%%%%%%%%%%%%%%%%%%%%%%%%%%%%%%%%%%%%%%%%%%%%%%%%%%%%%%%%%%%%%%%
\subsection*{Intracranial Encephalography (iEEG)}
%%%%%%%%%%%%%%%%%%%%%%%%%%%%%%%%%%%%%%%%%%%%%%%%%%%%%%%%%%%%%%%%%%%%%%%%%%%%%%%%%%%%
Hans Berger EEG
Nature of signal. Net activity of population – Locality, Number of cells, etc. Volume conduction. 
Medically intractable focal epilepsy. All patients are candidates for resective epilepsy surgery. Electrodes are implanted solely for the purpose of localizing.
Despite patients suffering from epilepsy, the majority physiological  
%[Cite: Origins of extracellular feilds - Buszaki, Anastassiou, & Koch, 2012]

%%%%%%%%%%%%%%%%%%%%%%%%%%%%%%%%%%%%%%%%%%%%%%%%%%%%%%%%%%%%%%%%%%%%%%%%%%%%%%%%%%%%
\subsection*{Network Connectivity}
%%%%%%%%%%%%%%%%%%%%%%%%%%%%%%%%%%%%%%%%%%%%%%%%%%%%%%%%%%%%%%%%%%%%%%%%%%%%%%%%%%%%
Functional connectivity involves characterizing the statistical relationship between brain signals, while anatomical connectivity is concerned with describing the way the physical components of the brain are physically connected. However, a mechanistic account of the brain involves describing how the anatomy causes its functional connectivity, and this so called effective connectivity consists of a model describing how the anatomical elements affect each other.

Probably the simplest way this could be accomplished is with a time dependent system of linear equations modeling the signals in each region as a function of the signals in the regions it is connected. In this sense, the method above is a measure of effective connectivity for local regions of the brain. Cortex is highly 

Contrast with other network connectivity measures, e.g. zero-lag correlation, phase-lag, etc. Lag threads.

“Therefore, the dynamical organization of rs-fMRI and its relation to brain states may manifest more fundamentally in spatiotemporal trajectories than changes in correlation structure.” (Mitra, 2018). “... our results suggest that—at least for the frequencies and regions we examined—the precise frequency of an oscillation could most closely relate to broad physiological factors such as the direction of wave propagation...The coexistence of traveling waves and CFC suggests that spatial bands of high-frequency neural activity move across the human cortex during behavior (Bahramisharif et al., 2013)” (Zhang, 2018)

%%%%%%%%%%%%%%%%%%%%%%%%%%%%%%%%%%%%%%%%%%%%%%%%%%%%%%%%%%%%%%%%%%%%%%%%%%%%%%%%%%%%
\subsection*{Traveling waves}
%%%%%%%%%%%%%%%%%%%%%%%%%%%%%%%%%%%%%%%%%%%%%%%%%%%%%%%%%%%%%%%%%%%%%%%%%%%%%%%%%%%%
\subsection*{Data preprocessing}
Re-referencing. Filtering. Artifact removal. SW and spindle detection.
Electrode localization. Surface mesh. 
\subsubsection*{Spatial embedding of electrodes}
Euclidean vs Geodesic distance. Electrode to mesh. Fast-marching algorithm. Transforming distance matrix to 3D coordinates.

\subsubsection*{Lag time network}
Peak lag. Lag of maximum correlation. 

%%% Local Variables: ***
%%% mode: latex ***
%%% TeX-master: "thesis.tex" ***
%%% End: ***
