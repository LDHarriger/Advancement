\mychapter{1}{Specific Aims}
%Memory formation, consolidation, and maintenance are fundamental functions of the brain.
%and likewise sleep dysfunction is closely linked to brain disorder. M
%Reflecting the clear changes in an animal's behavior during sleep, there are also major changes in brain physiology.
%Sleep serves a vital homeostatic role in maintaining brain health, and memory $-$ its formation, consolidation, and maintenance $-$ is a fundamental function of the brain, making it a strong index of brain health; 
%Memory is a fundamental function of the brain, thus an important index of brain health; supporting this health, sleep serves a vital homeostatic role in maintaining this health. 
%As a fundamental function of the brain, memory is an important index of brain health, and sleep serves a vital homeostatic role in maintaining this health.
Sleep serves a vital homeostatic role in maintaining brain health; and memory $-$ as a fundamental function of the brain $-$ is an important index of this health. Accordingly, sleep improves memory, while sleep deprivation impairs it; Alzheimers disease (AD) is a particularly striking example of this connection: one of the first side-effects of AD is forgetfulness, and accumulating evidence suggests that the disruption of homeostatic processes, which occur during sleep, is mechanistically linked to AD pathogenesis. These processes seem to occur during slow wave sleep (SWS), a deep phase of sleep when neuronal activity is modulated at low-frequency cycles ($\sim$1 Hz), called slow waves (SW), oscillating between active and inactive firing states. The SW is precisely coordinated with other waveforms like the spindle and sharp-wave ripple (SWR) in a way that promotes memory consolidation; potentially by reactivating previously learned neural patterns in a process called replay. Concurrent with slow-wave activity is a process called glymphatic clearance, during which cerebrospinal fluid (CSF) flows rapidly from para-arterial spaces into the brain's parenchyma and exits through paravenous spaces carrying toxic metabolites with it. Despite the importance of these sleep processes, replay and glymphatic clearance remain largely unexplored in humans, due primarily to a lack of adequate methods, which also satisfy the paramount ethical considerations for studying human subjects.
%disruptions related to dementia.
%The rhythms of electrical activity begin slow as neurons display bimodal firing patterns alternating between active and inactive states that are coincident with high power cycles in electric field called slow waves (SW).
% I propose combining structural, functional, and flow magnetic resonance imaging (MRI) as well as electroencephalography (EEG) to examine how electrical and fluid dynamics are organized in the brain to support sleep homeostasis, and determine how these dynamics are altered by memory, age, and dementia. 
%I propose examining how electrical dynamics are altered by memory, age, and dementia.
%combining structural, functional, and flow magnetic resonance imaging (MRI) as well as electroencephalography (EEG) and fluid dynamics are organized in the brain to support sleep
%While the majority of these techniques can be readily utilized with available resources,

Seeking evidence of these processes in humans, I propose using human intracranial recordings to examine how sleep electrophysiology supports memory consolidation, and additionally, extending the use of phase contrast magnetic resonance imaging (PCMRI) to quantify the effect of sleep on CSF flow. There are suggestions that cortical activity patterns are replayed in humans, but a strong experimental design and single unit data will greatly strengthen these claims. In contrast, there is hardly any evidence of glymphatic activity in humans; so to this end, I am actively piloting a PCMRI sequence to measure low-velocity flow of extracellular CSF. The feasibility of extending PCMRI for this application is supported by its routine use in measuring higher velocity flow of blood and CSF; although initial pilot data is encouraging, it is inconclusive. By utilizing several ongoing or imminent collaborations, I will have access to unique subject populations including: (1) epilepsy patients implanted with intracranial EEG (iEEG), (2) individuals aged 65-80 who may have pre-clinical signs of AD and will participate in MRI studies, (3) patients with sleep dysfunction who will participate in EEG and potentially MRI studies, and (4) healthy controls who will participate in MRI and EEG studies. Based on the estimated subject throughput associated with these collaborations, I should be able to collect sufficient data for these projects in about 1.5 years.

%The piloting phase for developing this technique should be completed by the end of Fall, at which point data collection may begin. already exists a large repository of iEEG recordings in subjects from (1) during sleep and wake, which is currently being utilized to address Aim 1, and for Aims 2 $\&$ 3, 

\textbf{\textit{Aim 1 will identify basic changes in brain network activity as a function of sleep state.}} Various waveforms in the brain have been observed to propagate in a spatially coherent manner $-$ referred to as traveling waves. Using CT and MRI, intracranial electrodes will be localized to their surrounding brain matter, and this spatial information will be combined with spontaneous sleep recordings to determine the extent of SW spatial propagation. I will develop a method to determine the pattern of this propagation, and also apply it to the spindle band to determine how SW and spindle network dynamics couple in space. 

\textbf{\textit{Aim 2 seeks to examine how sleep processes support biographical memories.}} An audio-visual name-face association task will be administered to subjects before and after sleep. I will identify brain network dynamics underlying cued memory reactivation to determine if these network dynamics are replayed during sleep, and if so, whether they enhance memory consolidation.  
%Memory performance will be correlated with various metrics of sleep quality and depth.
%in a variety of contexts including stimulus presentation, partial presentation (name or face), imagination, and the presentation of name audio clips during sleep.

\textbf{\textit{Aim 3 will extend PCMRI to quantify brain-wide CSF flow dynamics as a function of brain state.}} Studies in rodents demonstrate that, glymphatic CSF flux significantly increases during sleep, but is disrupted with age and in AD models. We will pilot the use of PCMRI to measure flow changes during wake vs sleep in effort to establish a translational link to the rodent work. Following pilot work, we will seek collaboration to validate this technique in rodents, and additionally assess the effect of age on on this flow.
%and additionally I evaluate whether this flow is disrupted in patients with AD. The piloting phase for developing this technique should be completed by the end of Fall, at which point data collection may begin.
% In rodents, glymphatic activity has been shown to substantially increase during SWS the promote the clearance of toxins like amyloid-beta, and therefore may be a mechanistic factor in Alzheimer's disease (AD). 

%%% Local Variables: ***
%%% mode: latex ***
%%% TeX-master: "thesis.tex" ***
%%% End: ***
