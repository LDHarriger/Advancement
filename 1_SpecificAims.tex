\mychapter{1}{Specific Aims}

% Aim 1: Identify changes in brain network activity as a function of sleep state.
% Aim 2: Examine sleep processes supporting biographical memories
% Aim 3: Develop MRI method to quantify brain-wide CSF flow dynamics as a function of brain state

The consequences of sleep deprivation are a growing list including loss of sex drive, elevated blood pressure, weakened immunity, cognitive impairment, and increased risk for a wide range of diseases. So while unconsciousness is the most obvious feature of sleep, it is just a side-effect of this profound state change which facilitates critical homeostatic processes throughout the body - especially in the brain. Around sleep onset, movement and sensory stimulation are reduced along with the demand for oxygen in tissues serving these functions, thus permitting respiration to slow and the heart relax. As the brain falls deeper into sleep, the concentration of neuromodulatory transmitters, like norepinephrine and serotonin, decrease, and neuronal activity alternates between active and inactive firing states modulated by $\sim$1 Hz waveforms called slow waves (SW). In parallel, astrocytes reduce their rate of lactate production, seek out synapses to phagocytose, alter ionic composition and permit greater flow of extracellular cerebrospinal fluid (CSF) through the brain. Importantly, these sleep processes and others have strong connections to cognition, aging, and disease, but some of these findings are inadequately studied in humans mostly limited by suitable methods that also meet the paramount ethical considerations for studying human subjects.
% Traveling waves evident in VSD, replay evident from dense micro-electrodes, and glymphatic activity.

One path forward is to combine existing techniques with novel analyses. I propose using combinations of structural, functional, and flow magnetic resonance imaging (MRI) as well as electroencephalography (EEG) to examine how fluid and electrical dynamics are organized by the brain to support sleep homeostasis, and determine how these dynamics are altered by learning, age, and dementia. While the majority of these techniques can be readily utilized with available resources, I am currently piloting a phase-contrast MRI (pcMRI) sequence to extend its application to the measurement of low-velocity extracellular CSF flow; however, the feasibility of this application is supported by the routine use of pcMRI to measure higher velocity flow of blood and CSF in larger structures as well as our initial pilot data. By utilizing several ongoing or imminent collaborations, I will have access to unique subject populations including: (1) epilepsy patients implanted with intracranial EEG (iEEG), (2) patients with varying degrees of dementia who will participate in MRI and potentially scalp EEG studies, (3) patients with sleep dysfunction who will participate in EEG and potentially MRI studies, and (4) healthy controls who would participate in MRI and EEG studies. There already exists a large repository of iEEG recordings in subjects from (1) during sleep and wake, which is currently being utilized to address Aim 1, and given subject throughput of our existing collaborations, I should be able to collect sufficient data for the remaining aims in about a year.

% Collecting adequate data should be possible within 1-2 years by leveraging 
%, integrating information from multiple modalities to contextualize data in a more natural and comprehensive space
\textbf{\textit{Aim 1 will identify basic changes in brain network activity as a function of sleep state.}} Intracranial electrodes will be localized to the brain using CT and MRI, and spontaneous recordings will be analyzed to assess how spatiotemporal dynamics of SW activity differ between sleep and wake. This will also be applied to the spindle band to further evaluate how SW and spindle networks couple in space and time. 

\textbf{\textit{Aim 2 is to examine how sleep processes support biographical memories.}} An audio-visual name-face association task will be administered to subjects before and after sleep. Memory performance will be correlated with various metrics for sleep quality and depth. Additionally, I will identify task relevant brain network dynamics in a variety of contexts including stimulus presentation, partial presentation (name or face), imagination, and the presentation of name audio clips during sleep. Finally, spontaneous sleep recordings will be analyzed for evidence of reactivation of these brain networks.      

\textbf{\textit{Aim 3 will extend existing MRI methods to quantify brain-wide CSF flow dynamics as a function of brain state.}} Flow of CSF through the brain parenchyma is regulated by the glymphatic system, which is formed by astrocytes. This glymphatic activity substantially increases during SWS promoting the clearance of toxins like amyloid-$\beta$, and therefore may be a mechanistic factor in Alzheimer's disease (AD).  To provide a translational link to this line of research, I propose using phase-contrast MRI to quantify changes in CSF flow between wake and sleep; additionally, I evaluate whether this flow is disrupted in patients with AD.  

%- which is routinely used to measure blood and CSF flow in large structures with velocities $>$10cm/s to show evidence of glymphatic activity in humans. 

% During sleep, animals lose awareness of the environment as well as their will to affect it, but this cost of losing of touch is accompanied by a provisional guarantee on the stability for the body: by drastically limiting its interaction with the environment, the body can refocus its energy on other needs, such as growth, repair, and cleanup. The value of this process might be appreciated in terms of the consequences of sleep deprivation - a long list including, loss of sex drive, weakened immunity, increased blood pressure, and cognitive impairment, to name as few. Around sleep onset, movement and sensory stimulation decrease, along with the demand for oxygen by the tissues serving these functions, thus permitting respiration to slow and the heart relax. As the brain falls deeper into sleep, neuronal activity is increasingly modulated at low-frequency cycles around 1 Hz called slow waves (SW) oscillating between inactive and active firing states. In parallel, glia, especially astrocytes undergo a major shift in activity; for example, altering ionic composition and permitting greater flow of interstitial fluid, reducing their rate of lactate production, increasing phagocytosis of damaged neural components.
%Importantly, during sleep as well as wake, neurons do not operate in isolation, but are supported by the entire body, and most intimately by its direct relatives the glia. Yet despite this critical dependence, studies of the human brain have a tendency to fixate on neural factors.
%While this dependence may be negligible on short-time scales 



%%


% Many has been discovered about sleep processes, but 
% integration, extend to humans, translational, 

%While cells do have a functional specialization, they are not isolated.
%However, periodically, sleep will shift into another state, in which neural activity more closely resembles wake. 
%there is a slowing in neural activity: among other changes, 
%discount the activity of non-neuronal tissue.
%[and becomes less interdependent]
%the body generally slows, for instance, 

%These slow waves are closely related to unconsciousness, but mental state is only one effect of sleep. 
%Due to special ethical considerations for studying human subject? Perhaps it is conceptually simpler? 
%The development of the brain is a story of neurons and glia eventually emerging together as daughters of the gastrula's neural crest cells. These cells organize themselves into a complex pattern harmonizing with the parallel development of the vascular system and overcoming space constraints of the skull by folding as it continues to expand. Of course, this development does not stop after birth, but continues, albeit in a different way, as the organism adapts an environment and body in constant change.

%Like the rest of nature, the life of an organism consists of persistent cycles interweaving with one another to support the whole.
%Sleep is a profound change in our state of consciousness, but also the functioning of our body.

%fMRI: Neural representation of 2D objects VS hemodynamic response to visual stimulation 
%to adopt a neuro-centric perspective, in which the non-nervous tissues 
% are considered in relative isolation, only interacting with each other
%place little emphasis on studying interaction of the nervous system. 

%Aim 2: Evaluate physiological responses supporting multi-sensory association task memory processes and during sleep.}
%reactivation in audio-visual paired-association task.}
%Develop phase-contrast magnetic resonance imaging methods to quantify glymphatic activity in human brain.}
%Effective network connectivity indicated by wave propagation patterns.

%%% Local Variables: ***
%%% mode: latex ***
%%% TeX-master: "thesis.tex" ***
%%% End: ***
